\documentclass[a4paper,11pt,fleqn, titlepage]{article}

\usepackage[utf8]{inputenc}
\usepackage[swedish]{babel}
\usepackage[lighttt]{lmodern}
\usepackage{parskip}
\usepackage{amsmath}
\usepackage{amssymb}
\usepackage{amsthm}
\usepackage{listings}
\usepackage{graphicx}

\author{Andreas Hagesjö \and Daniel Pettersson \and
Magnus Hagmar \and Niclas Ogeryd \and Robert Nyquist}

\title{Fossilfri bilflotta \\ Kurs ENM155}


\begin{document}
\maketitle

\section{Introduktion}
I denna rapport undersöks möjligheterna för att göra Sveriges bilflotta
fossiloberoende till år 2030. Just nu kommer den största delen av energin
som används i transportsektorn från just fossila bränslen. Detta betyder
såklart att väldigt mycket utsläpp orsakas av alla bilar, utöver den stora
energiförlusten på grund av den låga verkningsgraden för
förbränningsmotorer. Genom att bryta beroendet utav fossila bränslen
förbättras miljön alltså både genom minskade utsläpp och energiförbrukning.

Denna frågeställning är intressant att studera eftersom regeringen har satt
just detta målet för Sverige. Utöver detta så är det både ett aktuellt ämne
i samhället samt att det enligt flera utförda studier går att uppfylla
målsättningen. Denna rapport innehåller en modell som visar vilka energier
som kan användas för att nå detta mål samt olika scenarier för hur målet
kan uppnås. För att det skall vara möjligt så krävs det vissa
ansträngningar och tekniska utvecklingar.

\subsection{El}

För personbilar så kommer eldrift att ersätta stor del av bensin- och
dieseldrift. Då elbilar är effektivare än bilar med förbränningsmotorer så
kommer bytet till elbilar medföra att mängden energi som behöves för att
driva bilflottan att minska\footnote{Ett fossilbränsleoberoende
transportsystem år 2030 – Ett visionsprojekt för Svensk Energi och Elforsk,
sida 25}. Då stor del utav elbilarna som säljs idag är laddhybrider,
Mitsubishi Outlander står ensam för nästan 30 procent av det laddbara
beståndet i Sverige, så kommer den typen av bilar fortfarande rulla 2030
och därmed fortfarande använda fossilt bränsle. \\
\footnote{Ökningen av laddbara fordon avtar,
 2014-12-03, \\
http://powercircle.org/nyhet/okningen-av-laddbara-fordon-avtar-3 (hämtad
2015-01-06)}.
I nuläget så är det endast
7800 av de cirka 4.5 miljoner bilarna i sverige som kan drivas på el
\footnote {Fordonsstatistik 2014, Transportstyrelsen. \\
https://www.transportstyrelsen.se/sv/Press/Statistik/Vag/Fordonsstatistik/
Fordonsstatistik-juni-2014/ }. Det kommer att diskuteras huruvida 
utvecklingen kan ske till 2030 för att se till att eldrivna bilar kommer
att bli en mer betydande del av fordonsflottan gentemot hur det ser ut
idag, och att efterfrågan för elbilarna kommer att öka drastiskt.
Regeringen har i nuläget infört styrmedel för att gynna elbilar
(och de absolut bästa gasbilarna), med en supermiljöbilspremie, som innebär
att om en bils maximala koldioxidsläpp per kilometer är 50 gram, så kan denna
premie tilldelas\footnote{Ekonomiska Styrmedel, \\
http://www.biogasportalen.se/BliProducentAvBiogas/Ekonomi/Stodochstyrmedel}

En fråga man kan ställa sig är huruvida det kommer gå att producera den
extra mängden el som behöves för att utöka beståndet med elbilar. Som det
ser ut nu så skulle bilflottan behöva runt 53 TWh/år år 2030 utan några
åtgärder\footnote{Ett fossilbränsleoberoende transportsystem år 2030 – Ett
visionsprojekt för Svensk Energi och Elforsk, tabell 3 sida 24}, men med
den effektiviseringen som kommer med elmotorer, mellan 2,5 och 3 gånger så
effektiv\footnote{Ett fossilbränsleoberoende transportsystem år 2030 – Ett
visionsprojekt för Svensk Energi och Elforsk, sida 29}, så kommer den
mängden energi att minska så pass mycket att det inte blir nödvändigt med
så stor ökning utav elproduktionen.

\subsection{Biobränsle}
Tunga fordon så som lastbilar och maskiner kommer antagligen inte kunna
köras med eldrivna motorer inom de nästkommande åren, detta på grund av att
man idag inte kan lagra den mängden energi som behövs för att driva ett
stort och tungt fordon. Och eftersom de flesta tunga fordon idag körs på
diesel, kan ett bränslebyte till biodiesel gå relativt smärtfritt.

Samma sak är det med bensin och dieselbilar som säljs idag, eftersom
medelåldern på en personbil i Sverige är 9 år\footnote{pls insert
källa} så betyder det att det kommer köras bilar år 2030 som är sålda idag,
och då måste vi ha ett sätt att med så låg kostnad som möjligt, konvertera
dessa till fossilfria alternativ. Där ligger biobränslen närmast, en bil
som idag körs på diesel, kan som tidigare sagt, relativt smärtfritt köras
på biodiesel istället\footnote{källa på biodiesel i dieselbil}. Lite
svårare bil det däremot med en bensindriven bil. Där kan man bli tvungen
att göra lite större ändringar i motorn, vilket självklart resulterar i en
högre kostnad för konverteringen, men det är mycket möjligt att göra en sån
konvertering. Detta betyder att bilar som säljs idag och några år framåt,
som är avsedda att köras fossila bränslen som bensin och diesel, kan
konverteras och då framföras oberoende av fossila bränslen.

För att främja biobränslet i Sverige har man i nuläget infört ekonomiska styrmedel 
genom att låta allt biobränsle genom att låta dess skatt vara
avdragsgill.
\footnote{Ekonomiska styrmedel, \\
http://www.biogasportalen.se/BliProducentAvBiogas/Ekonomi/Stodochstyrmedel}.

Man får dock vara försiktig med just biobränslen, det finns inte
biobränslen i överflöd, och skulle andra länder få en ökad efterfrågan på
biobränslen kan det få priset att rusa i taket. Detta kan medföra stora
negativa effekter så som \emph{land-grabbing}\footnote{Bioenergi, Oskar
Englund, Energi och Miljö Fysisk resursteori, Chalmers.}, ökade matpriser i
fattigare länder etc. vilket man då kan reglera genom att man har kvar
möjligheten att tillfälligt köra på fossila bränslen, för att minska
Sveriges efterfrågan på biobränslen dels ur bränslesäkerhet men även
miljömässigt och kanske även ur ett etiskt perspektiv.

\subsection{Vätgas}
Bilar som drivs på vätgas ligger för tillfället bakom både elbilar och bilar som drivs på biobränsle och därför räknar vi med att år 2030 så består inte stora delar utav bilflottan av bilar som drivs av vätgas. Bilarna produceras fortfarande i små mängder och i sverige finns endast en laddningsstation för tillfället\footnote{Första macken för vätgas, 2014-06-10, \\
http://www.skanskan.se/article/20140610/MALMO/140609539/-/forsta-macken-for-vatgas (hämtad 2015-01-06)}.
Bilarna är dessutom fortfarande för dyra för att få ett stort genomslag än så länge. Bilar som drivs på vätgas har har stor potential i framtiden men 2030 så kommer de inte finnas i lika stor mäng som elbilar och biobränsle bilar.

\subsection{Kärnkraft}
Regerigen har upphävt förbudet mot ny kärnkraft i Sverige. Detta har lett till intresse till ersätta gamla reaktorer med nya, bland annat i Oskarshamn, som kan producera mer el.
Dessutom så kan man räkna med att samtliga aktiva reaktorer i Sverige kommer att producera en större mängd el om 15 år än vad de gör nu\footnote{Kärnkraft - nya reaktorer och ökat skadeståndsansvar, SOU 2009:88  tabell 2.2 sida80}. Därför räknar vi med att kärnkraften år 2030 har möjlighet att täcka upp för det ökade behovet utav elproduktion som en ökad mängd elbilar medför.

\section{Metod}
Vi har utgått från den angivna modellen i tidigare uppgift. Med den som
ursprungsläge samt information från flera rapporter och prognoser i ämnet
har vi kunnat få fram en hypotes till hur energisystemet kan komma att se
ut 2030. Vi har sen matat vår algoritm med hypotesen, den har sen räknat ut
resultatet och visar det i den form som vi behöver.

\section{Resultat}

\subsection{Scenario 1}

Det största användingsområdet för biomassa är i nuläget uppvärmning, om man istället använde
I Sverige finns det stora mängder mark som kan användas för att odla biomassa. En kraftigt ökad produktion, i samband med ett utbyte av fossila bränslen och biodrivmedel för uppvärmning möjliggör en ökning av bio-fordon. Existerande fordon kan även utrustas med teknologi som gör att de kan drivas med biobränslen.

\subsection{Scenario 2}

En stor del av uppvärmningen sker idag genom biovärme, om mängden biomassa som används för uppvärmning istället används inom produktion av biodrivmedel. Uppvärmingen skall istället ske med hjälp av el, detta innebär att elproduktionen måste ökas markant. För att tillgodose dessa energibehov ska kärnkraften utökas. Sverige har stora mängder uran som skulle användas som bränsle, dessutom ger kärnkraft stora mängder energi för en relativt liten mängd bränsle. På grund av de stora mängderna el som produceras så kommer det även finnas tillräckligt mycket för att ladda en ökad mängd elbilar.


\section{Diskussion}

\subsection{Scenario 1}

Själva bytet mellan användningsområdet för biomassa och fossil energi löser inga problem då den fossila energi fortfarande kommer användas men inom andra områden. Dock så utvecklas och effektiviseras Sveriges elproduktion konstant. Man räknar med att den är klimatneutral till 2050. Så mångden fossil energi som byts från bilar till uppvärmning kommer ändå fasas ut och sluta användas. Fossila btänslen och biobränslen har dessutom samma verkningsgrad när de används för värmeproduktion så ett byte ger ej några extra förluster för elproduktionen.  Elproduktionen är lättare att få fossilfri än bilflottan så är det ett vettigt alternativ att gör utbyte mellan dessa.

\subsection{Scenario 2}
Kärnkraft kommer med tiden att bytas ut mot förnyelsebar energi vilket kommer ta längre tid om vi blir mer beroende av det på grund utav en ökad mängd elbilar. Detta är dock inget större problem då själva övergången fortfarande kommer att vara genomförbar.



\subsection{Båda scenarier}
Världen har gått och går fortfarande igenom stora ekonomiska problem så får inte övergången till en fossilfri bilflotta i sverige fresta ekonomin för mycket. Frestar övergången på personlig ekonomi och statlig ekonomi så kommer det ta längre tid samt bli svårare att genomföra.


Efterfrågan på biomassa kan ge stor påverkan på priset på mat och mark att odla på så behöver man vara försiktig med hur mycket man ökar efterfrågan. Ska man öka mängden bilar som använder biobränsle så är det viktigt att se till att det inte resulterar i negativa effekter på andra områden. 

\subsection{Scenario vätgas}
I litteraturen så har vätgasbilar väldigt liten inverkan på bilflottan 2030 (KOLLA DETTA), dessutom så har finns det flera större rapporter där alla pekar på att antalet vätgasbilar kommer vara marginell.
Då utvecklingen utav vätgasbilar fortfarande är i ett stadie där det experimenteras mycket och modellerna endast finns i små mängder så har de ett väldigt högt inköpspris. För att vätgasbilar skall finnas i en större utbredning år 2030 så måste bilföretag välja att börja producera i större skalor så att priserna går ner.

Statliga medel som tvingar fram tankstationer är också ett måste då det för tillfället endast finns en tankstation i Malmö och två planerade, en vadera i Stockholm och Göteborg.
Ekonomiska medel från staten skulle göra att försäljningen kan öka tidigare då försäljningspriset kan vara högra utan att direkt påverka privatpersoners ekonomi.
Tyskland, Japan och Karlifonien är de platser där uppbyggnaden utav vätgasstationer har kommit längst. Där har myndigheter sammarbetat med företag för att dela på kostnader och risker som det innebär. Sådana sammarbeten skulle även behövas i Sverige om man snabbt vill bygga upp en infrastruktur som stödjer vätgasbilar.

Något som talar för att utvecklingen för vätgasbilar skall gå framåt i snabbt takt är att det forskars på att använda vätgas som lagring för energi från exempelvis vindkraftverk. Detta skulle göra vindkraft mindre beroende av andra energikällor när det inte blåser. Utvecklingen av att binda energi i vätgas har alltså fördelar i andra områden än transport vilket gör att forskning inom området är lönsamt för flera. Det finns även industrier som har vätgas som spillvara som man kan ta vara å för att få vätgas direkt utan några energiförluster vid tillverkningen.


\end{document}

