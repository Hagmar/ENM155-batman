\documentclass[a4paper,11pt,fleqn, titlepage]{article}

\usepackage[utf8]{inputenc}
\usepackage[swedish]{babel}
\usepackage[lighttt]{lmodern}
\usepackage{parskip}
\usepackage{amsmath}
\usepackage{amssymb}
\usepackage{amsthm}
\usepackage{listings}
\usepackage{graphicx}

\author{Andreas Hagesjö \and Daniel Pettersson \and
Magnus Hagmar \and Niclas Ogeryd \and Robert Nyquist}

\title{Fossilfri bilflotta \\ Kurs ENM155} 


\begin{document}
\maketitle

\section{Introduktion}
I denna rapport undersöks möjligheterna för att göra Sveriges bilflotta
fossiloberoende till år 2030. Just nu kommer den största delen av energin
som används i transportsektorn från just fossila bränslen. Detta betyder
såklart att väldigt mycket utsläpp orsakas av alla bilar, utöver den stora
energiförlusten på grund av den låga verkningsgraden för
förbränningsmotorer. Genom att bryta beroendet utav fossila bränslen
förbättras miljön alltså både genom minskade utsläpp och energiförbrukning.

Denna frågeställning är intressant att studera eftersom regeringen har satt
just detta målet för Sverige. Utöver detta så är det både ett aktuellt ämne
i samhället samt att det enligt flera utförda studier går att uppfylla
målsättningen. Denna rapport innehåller en modell som visar vilka energier
som kan användas för att nå detta mål samt olika scenarier för hur målet
kan uppnås. För att det skall vara möjligt så krävs det vissa
ansträngningar och tekniska utvecklingar.

\section{Metod}

Vår modell visar de energier som är mest troliga att ersätta fossila.
Modellen är byggd så att det enkelt går att skifta mellan olika scenarier
där olika energier kommer ha olika stort bidrag. I våra scenarion så utgår
vi ifrån att effektivisering av bilflottan och att minskning av själva
transportbehovet blir mycket framgångsrikt.


Flödesschemat i Appendix A visar hur implementationen för att visa de olika
scenarierna.

\subsection{El}

För personbilar så kommer eldrift att ersätta stor del av bensin- och
dieseldrift. Då elbilar är effektivare än bilar med förbränningsmotorer så
kommer bytet till elbilar medföra att mängden energi som behöves för att
driva bilflottan att minska\footnote{Ett fossilbränsleoberoende
transportsystem år 2030 – Ett visionsprojekt för Svensk Energi och Elforsk,
sida 25}. Då stor del utav elbilarna som säljs idag är laddhybrider,
Mitsubishi Outlander står ensam för nästan 30 procent av det laddbara
beståndet i Sverige, så kommer den typen av bilar fortfarande rulla 2030
och därmed fortfarande använda fossilt bränsle\footnote{Ökningen av
laddbara fordon avtar, 2014-12-03, \\
http://powercircle.org/nyhet/okningen-av-laddbara-fordon-avtar-3 (hämtad
2015-01-06)}.

En fråga man kan ställa sig är huruvida det kommer gå att producera den
extra mängden el som behöves för att utäka beståndet med elbilar. Som det
ser ut nu så skulle bilflottan behöva runt 53 TWh/år år 2030 utan några
åtgärder\footnote{Ett fossilbränsleoberoende transportsystem år 2030 – Ett
visionsprojekt för Svensk Energi och Elforsk, tabell 3 sida 24}, men med
den effektiviseringen som kommer med elmotorer, mellan 2,5 och 3 gånger så
effektiv\footnote{Ett fossilbränsleoberoende transportsystem år 2030 – Ett
visionsprojekt för Svensk Energi och Elforsk, sida 29}, så kommer den
mängden energi att minska så pass mycket att det inte blir nödvändigt med
så stor ökning utav elproduktionen.

\subsection{Biobränsle}
Tunga fordon så som lastbilar och maskiner kommer antagligen inte kunna
köras med eldrivna motorer inom de nästkommande åren, detta på grund av att
man idag inte kan lagra den mängden energi som behövs för att driva ett
stort och tungt fordon. Och eftersom de flesta tunga fordon idag körs på
diesel, kan ett bränslebyte till biodiesel gå relativt smärtfritt.

Samma sak är det med bensin och dieselbilar som säljs idag, eftersom
medellivslängden på en personbil i Sverige är 9 år\footnote{pls insert
källa} så betyder det att det kommer köras bilar år 2030 som är sålda idag,
och då måste vi ha ett sätt att med så låg kostnad som möjligt, konvertera
dessa till fossilfria alternativ. Där ligger biobränslen närmast, en bil
som idag körs på diesel, kan som tidigare sagt, relativt smärtfritt köras
på biodiesel istället\footnote{källa på biodiesel i dieselbil}. Lite
svårare bil det däremot med en bensindriven bil. Där kan man bli tvungen
att göra lite större ändringar i motorn, vilket självklart resulterar i en
högre kostnad för konverteringen, men det är mycket möjligt att göra en sån
konvertering. Detta betyder att bilar som säljs idag och några år framåt,
som är avsedda att köras fossila bränslen som bensin och diesel, kan
konverteras och då framföras oberoende av fossila bränslen.

Man får dock vara försiktig med just biobränslen, det finns inte
biobränslen i överflöd, och skulle andra länder få en ökad efterfrågan på
biobränslen kan det få priset att rusa i taket. Detta kan medföra stora
negativa effekter så som \emph{land-grabbing}\footnote{Bioenergi, Oskar
Englund, Energi och Miljö Fysisk resursteori, Chalmers.}, ökade matpriser i
fattigare länder etc. vilket man då kan reglera genom att man har kvar
möjligheten att tillfälligt köra på fossila bränslen, för att minska
Sveriges efterfrågan på biobränslen dels ur bränslesäkerhet men även
miljömässigt och kanske även ur ett etiskt perspektiv.

\section{Resultat}

\end{document}

