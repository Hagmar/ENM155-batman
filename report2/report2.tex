\documentclass[a4paper,11pt,fleqn, titlepage]{article}

\usepackage[utf8]{inputenc}
\usepackage[swedish]{babel}
\usepackage[lighttt]{lmodern}
\usepackage{parskip}
\usepackage{amsmath}
\usepackage{amssymb}
\usepackage{amsthm}
\usepackage{listings}
\usepackage{graphicx}

\author{Andreas Hagesjö \and Daniel Pettersson \and
Magnus Hagmar \and Niclas Ogeryd \and Robert Nyquist}

\title{Fossilfri bilflotta \\ Kurs ENM155} 


\begin{document}
\maketitle

\section{Introduktion}
Regeringen har ett mål om att göra Sveriges bilflotta fossiloberoende till 2030. Denna rapport innehåller en modell som visar vilka energier som kan användas för att nå detta mål samt olika scenarier för hur målet kan uppnås.
För att det skall vara möjligt så krävs det vissa ansträngningar och tekniska utvecklingar.


\section{Metod}

Våran modell visar de energier som är mest troliga att ersätta fossila. Modellen är byggd så att det enkelt går att skifta mellan olika scenarier där olika energier kommer ha olika stort bidrag.
I våra scenario så utgår vi ifrån att effektivisering av bilflottan och att minskning av själva transportbehovet blir mycket framgångsrikt.


Flödesschemat i Appendix A visar hur implementationen för att visa de olika scenarierna.

\subsection{El}

Stor del utav bilflottan drivs på el. Då elbilar är effektivare än bilar med förbränningsmotorer så kommer bytet till elbilar medföra att mängden energi som behöves för att driva bilflottan att minska. (REFERENS! TILL REGERINGSRAPPORT!)

\section{Resultat}

\end{document}

