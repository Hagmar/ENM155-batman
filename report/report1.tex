\documentclass[a4paper,11pt,fleqn]{article}

\usepackage[utf8]{inputenc}
\usepackage[swedish]{babel}
\usepackage[lighttt]{lmodern}

%\usepackage{amsmath}
%\usepackage{amssymb}
%\usepackage{amsthm}
%\usepackage{listings}
%\usepackage{parskip}
%\usepackage{enumerate}
%\usepackage{tikz}
%\usepackage{graphicx}

%\usepackage{geometry}
%\geometry{
%	%top=1.5in,
%	bottom=1.5in
%}

\author{Andreas Hagesjö \and Daniel Pettersson \and
Magnus Hagmar \and Niclas Ogeryd \and Robert Nyquist}

\title{Modell över Sverige primärenergitillförsel \\ Kurs ENM155}

\begin{document}
\maketitle

\section{Introduktion}
Denna rapport innehåller en enkel modell utav Sveriges energisystem som det ser ut idag.
Den innehåller en matematisk modell samten uppskattning utav Sveriges totala primärenergitilförsel.


\section{Metod}
Beskriv hur vi löst uppgiften, genom att beskriva vår modell, dels
matematiskt och dels med ord. Utöver den tekniska beskrivningen ska vi
svara på följande konkreta frågor om vår modell:

\begin{itemize}
\item Följer er implementationsstruktur strukturen på diagrammet i Figur 1?
I så fall hur? Om inte, finns det något liknande flödesschema som mer
direkt motsvarar er implementation?

\item Ge ett par exempel på hur man kan utvinna intressant information från
er modell. Är den användbar för något särskilt utöver huvuduppgiften (att
beräkna total primärenergitillförsel)?

\item Beskriv kortfattat hur man skulle gå till väga för att lägga till ett
nytt primärenergislag i er modell (till exempel solenergi). Samma sak för
ett nytt sekundärenergislag (till exempel vätgas genom elektrolys och
reformering).
\end{itemize}

\section{Resultat}
Presentera Sveriges totala primärenergitillförsel, samt uppdelat på
respektive energikälla.

\appendix
\section{Programkod}
Bifoga koden

\end{document}

